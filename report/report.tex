\documentclass[czech]{article}

\usepackage[utf8]{inputenc}
\usepackage[IL2]{fontenc}
\usepackage[czech]{babel}
\usepackage[a4paper,textheight=674pt]{geometry}
\usepackage{hyperref}
\usepackage{graphicx}

\begin{document}
\begin{center}\large
\bf Semestrální projekt MI-PDP 2017/2018:\\[6mm]
    Paralelní algoritmus pro řešení problému\\[3mm]
    Bílá královna na šachovnici -- KAS\\[6mm]
    Josef Doležal\\[2mm]
    magisterské studium, FIT ČVUT, Thákurova 9, 160 00 Praha 6\\[2mm]
    \today
\end{center}

\thispagestyle{empty}
\newpage

\section{Definice problému}

Problém bílé královny na šachovnici je nalezení minimálního počtu tahů, které musí královna udělat, aby vzala všechny oponentovi pěšce.
Tento problém je zadán jako šachovnice (o velikosti $n$), kolekce souřadnic pěšců a pozice královny.
Má-li úloha řešení, pak horní mez pro sebrání pěšců je rovna $3 \cdot q$, kde $q$ je počet pěšců.
Tato horní mez je mnohdy výrazně vyšší než skutečné řešení, z tohoto důvodu program očekává dodatečnou informaci o horním odhadu.

Složitost tohoto algoritmu při řešení hrubou silou je (doplnit).
Pro vstupy o malých rozměrech šachovnice je možné problém tímto způsobem řešit.
Časová složit ale roste exponencionálně s velikostí vstupu, pro velké $n$ je tak možné, že program nenalezne řešení v konečném čase.

Pro vstupy o velikosti 10 šachovnicových polí už doba běhu může přesáhnout jednotky minut.
Je-li vstup nepatrně větší (např. 15 polí), program už nedoběhne v přijatelné době.

Tento problém je tedy vhodný na využití paralelního výpočtu, jehož zkoumáním se práce zabývá.

\section{Popis paralelního algoritmu a jeho implementace v OpenMP}

\section{Popis paralelního algoritmu a jeho implementace v MPI}

\section{Naměřené výsledky a vyhodnocení}

\section{Závěr}

\section{Literatura}

\appendix

\end{document}